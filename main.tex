\documentclass[a4paper,11pt]{article}
\usepackage[english]{babel}
\usepackage{hyperref}{}

\begin{document}

\title{Guidelines for Software Quality}
\subtitle{CLARIAH Task 54.100} 

\author{Maarten van Gompel}

\maketitle

\section{Introduction}

CLARIAH aims to deliver a digital research infrastructure made explicitly
accessible for researchers from the humanities. This makes the development of advanced ICT
tools a core activity within CLARIAH. To be able assess the
quality of the research infrastructure as a whole, we need to be able to assess
the quality of its individual software and data components. If we can establish
a common set of software guidelines, we may more readily identify weaker
components of the software infrastructure and work on their improvement.
Assessing software or data quality, however, is not a trivial matter. 

Whenever we refer to \emph{software}, we intend the term in a broad sense and
encompassing all of the following aspects:
\begin{itemize}
    \item source code
    \item binary executables
    \item user interfaces (including application programming interfaces (APIs) and web APIs)
    \item associated essential data
    \item documentation (including tutorials, screencasts)
    \item support infrastructure (version control, build systems, issue trackers)
\end{itemize}

The guidelines for software quality will be formulated as a series of
assessment criteria, posed as questions, divided over several categories. This
makes them directly applicable as an instrument for software quality
assessment.

The criteria we yield, and their categories, are derived to large extent from
the \emph{criteria-based software evaluation guide} by the Software
Sustainability Institute. Their work, in turn, is modelled after \emph{ISO/IEC
9126-1 Software Engineering - Product Quality}.  The Software
Sustainability Institute is an academic institute explicitly geared towards
researchers and software developers in science, and as-such its work is of
great relevance to projects such as CLARIAH.







\end{document}
