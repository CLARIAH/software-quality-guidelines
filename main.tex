\documentclass[a4paper,11pt]{article}
\usepackage[english]{babel}
\usepackage{hyperref}{}

\begin{document}

\title{Guidelines for Software Quality}
\subtitle{CLARIAH Task 54.100} 

\author{Maarten van Gompel}

\maketitle

\section{Introduction}

CLARIAH aims to deliver a digital research infrastructure made explicitly
accessible for researchers from the humanities. This makes the development of advanced ICT
tools a core activity within CLARIAH. To be able assess the
quality of the research infrastructure as a whole, we need to be able to assess
the quality of its individual software and data components. If we can establish
a common set of software guidelines, we may more readily identify weaker
components of the software infrastructure and work on their improvement.
Assessing software or data quality, however, is not a trivial matter. 

Whenever we refer to \emph{software}, we intend the term in a broad sense and
encompassing all of the following aspects:
\begin{itemize}
    \item source code
    \item binary executables
    \item user interfaces (including application programming interfaces (APIs) and web APIs)
    \item associated essential data
    \item documentation (including tutorials, screencasts)
    \item support infrastructure (version control, build systems, issue trackers)
\end{itemize}

The guidelines for software quality will be formulated as a series of
assessment criteria, posed as questions, divided over several categories. This
makes them directly applicable as an instrument for software quality
assessment. These guidelines target developers, managers and users of software.
Developers will be more aware of the targets to meet, and able to identify and
remedy weak areas. Managers and users will be able to assess whether software is of
sufficient enough quality for their purposes. Although we intend to formulate
the questions as plainly as possible, a certain degree of technical
expertise is demanded of all of these in otder to attain a successful
assessment.

Each question, if applicable, should be answered affirmatively. This will lead
to measurable quality levels. Full compliance with the guidelines is achieved
when all questions are answered affirmatively. In practise, however, this is
not realistic to happen. Evaluators will need to determined for themselves what
compliance level, per category, constitutes an acceptable passing threshold.

The criteria we yield, and their categories, are derived to large extent from
the \emph{criteria-based software evaluation guide} \cite{SSIGUIDE} by the Software
Sustainability Institute (SSI). Their work, in turn, is modelled after \emph{ISO/IEC
9126-1 Software Engineering - Product Quality}.  The Software
Sustainability Institute is an academic institute explicitly geared towards
researchers and software developers in science, and as-such its work is of
great relevance to projects such as CLARIAH. 

The SSI, following ISO 9126-1, has grouped assessment criteria as follows
\cite{SSIGUIDE}.

\begin{tabular}{|l|l|}
\textbf{Usability} \\
Understandability & Is the software easily understood? \\ 
Documentation & Comprehensive well-structured documentation? \\
Buildability  & Straightforward to build from source on a supported system? \\
Installability  & Straightforward to install and deploy on a supported system? \\
Learnability & Easy/intuitive to learn how to use its functions? \\
\hline
\textbf{Sustainability \& Manageability}
Identity & Project/software identity is clear and unique? \\
Copyright & Easy to see who owns the project/software? \\
Licencing & Adoption of appropriate licence?
Governance & Easy to understand how the project is run and the development managed? \\
Community & Evidence of current/future community? \\
Accessibility & Evidence of good facilities to obtain versions of the software? \\
Testability & Easy to verify if the software functions correctly? \\
Portability & Usable on multiple platforms? \\
Supportability & Evidence of current/future developer support? \\
Analysability & Easy to understand at the source-code level? \\
Changeability & Easy to modify and contribute changes? \\
Evolvability & Evidence of current/future development? \\
Interoperability & Interoperable with other required/related software? \\
\end{tabular}

The next section will list all assessment criteria. %TODO: say something about difference  with SSI

\section{Quality Assessment Criteria - Usability}

\subsection{Understandability}

\subsubsection{U1 -- Is it clear what the software does?}

Software must be accompanied be a clear high-level description, describing what
exactly it does. Both the README file that ships with the software as well as
the project website should contain this information. 

\subsubsection{U2 -- Is it clear for whom the software is intended?}

It should be clear who are the intended users for the software. Software is
usually not appropriate for all audiences. Gearing software at multiple
audiences however, through for instance offering multiple interfaces (command
line, GUI, webservice) is good practise. References to projects already using the
software are suggested.

\subsubsection{U3 -- Is it clear how the software works?}

There should be a high-level description explaining how the software
accomplishes its task. Links to publications are recommended. Also, a schema
offering an architectural overview is suggested where appropriate. 

\subsubsection{U4 -- Is the software motivated?}

There should be a written motivation for why the software does things the way
it does and why it was designed in the first place. It should be clear what
problems are solved by it. Links to publications are suggested.

\subsection{Documentation}

When we refer to documentation, we refer to the set of all documentation
available for the software. This may consists of different types of
documentation for different audiences, and may include academic papers.

\subsubsection{DOC1 -- Is there documentation?}

All software should be properly documented. Software without any documentation
is as good as useless. At the very least, documentation at a minimum level
must be available. 

\subsubsection{DOC2 -- Is the documentation accessible?}

Documentation must be publicly accessible and in an acceptable standard format
such as HTML or PDF.

\subsubsection{DOC3 -- Is the documentation clear?}

Is it clear enough? Does it clearly describe the software. Step-by-step
and task-oriented instructions are recommended.

\subsubsection{DOC4 -- Is the documentation complete?}

Documentation should cover the entire software, including advanced features.

\subsubsection{DOC5 -- Is the documentation accurate?}

Documentation should describe the advertised version and not be out-of-date
with the latest release. Example should be in line with how the tool looks and
behaves.

\subsubsection{DOC6 -- Does the documentation provide a high-level overview of the software?}

Documentation should offer a high-level overview of the software, rather
than immediately dive into the details.

\subsubsection{DOC7 -- Does the address the necessary audiences, at their appropriate levels?}

Different groups of users require different documentation. Developers require
APIs if the software is a library, end-users require a walkthrough of the GUI
if the software has one. A different level of expertise may be expected of
different user groups, the documentation should assume the appropriate level.

\subsubsection{DOC8 -- Does the documentation make use of adequate examples?}

Documentation should contain examples appropriate for the interface that is
described. Command-line interfaces should see examples of invocation and input
and output. Graphical user interfaces should be illustrated through screenshots or
screencasts. API references should contain source code examples of usage.

\subsubsection{DOC9 -- Is there adequate troubleshooting information?}

Documentation should include information on troubleshooting, i.e.\ a
specification of possible error messages and explanation for resolution.

\subsubsection{DOC10 -- Is the documentation available from the project website?}

Documentation must be clearly linked from the project website.

\subsubsection{DOC11 -- Is the documentation under version control?}

The sources for documentation must be under version control like the source
code, preferably alongside the code itself.


\subsection{Buildability}

Buildability applies to all software written in languages that compile to
either native machine code or intermediate byte code to be interpeted by a VM.
This can be constrasted to software that is interpreted at run-time from source
code. This section is therefore applicable only to languages such as C, C++,
Java, Pascal, Haskell, Scala, Rust, Cython but not to scripting language such
as Python, Ruby, Perl, Go. 


\subsubsection{BLD1 -- Are there good instructions for building/compiling the software?}

Build/compilation instruction should be available and clear enough. They should
be distributed alongside the software's source distribution (as part of an
INSTALL file or README), and/or be addressed in the documentation. If the
source distribution is the primary means of distribution, then build
instructions should be prominently displayed on the project website as well.

\subsubsection{BLD2 -- Is an established automated build system used?}

Established build systems should be used. For example the GNU Build
System\footnote{also known as the autotools} or CMake for C/C++; Ant or Maven
for Java. 

\subsubsection{BLD3 -- Are all the third-party dependencies listed?}

All required or optional third-party dependencies should be listed (with
references to their websites). If the build system supports automatically
obtaining dependencies, then this is a preferred solution. If the platform has
a package manager that can install these distributions, then add instructions
(i.e, package names) to accomplish this.

Moreover, all listed dependencies should be available, unobtainable software
can not be used as a dependency. Higher-order dependencies need not be listed.

\subsubsection{BLD4 -- Are there tests to verify the build has succeeded?}


\subsection{Installability}
















\section{Quality Assessment Criteria - Sustainability \& Maintenance}




\end{tabular}







\section{Software Quality Assem}





\end{document}
